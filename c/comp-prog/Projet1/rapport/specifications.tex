% !TEX root = ./main.tex
%%%%%%%%%%%%%%%%%%%%%%%%%%%%%%%%%%%%%%%%%%%%%%%%%%%%%%%%%%%%%%%%%%%%%%%%%%%%%%%%%%%%%%%%%%
% Dans cette section, spécifiez formellement chacun des sous-problèmes.                  %
%%%%%%%%%%%%%%%%%%%%%%%%%%%%%%%%%%%%%%%%%%%%%%%%%%%%%%%%%%%%%%%%%%%%%%%%%%%%%%%%%%%%%%%%%%
\section{Specifications}\label{specifications}
%%%%%%%%%%%%%%%%%%%%%%%%



\subsection{SP1: Recherche du plus grand prefixe-suffixe:}
\begin{itemize}
    \item \textbf{Précondition}: $T$ pointer vers un tableau de longueur $N \land N \geq 0$
    \item \textbf{Postcondition}: $T = T_0 \land N = N_0$
    \item \textbf{Retour}: $ \operatorname{prefixe\_suffixe}(T, N) $
\end{itemize}



\subsection{SP2: Vérification que le préfixe et sufixe de longueur $k$ sont égaux:}
\begin{itemize}
    \item \textbf{Précondition}: $T$ pointer vers un tableau de longueur $N \land 0 < N \land < k < N$
    \item \textbf{Postcondition}: $T = T_0 \land N = N_0$
    \item \textbf{Retour}: $ \operatorname{pref\_equals\_suff}(T, N, k) $
\end{itemize}
