% !TEX root = ./main.tex
%%%%%%%%%%%%%%%%%%%%%%%%%%%%%%%%%%%%%%%%%%%%%%%%%%%%%%%%%%%%%%%%%%%%%%%%%%%%%%%%%%%%%%%%%%
% Dans ce fichier, vous devez définir (Input/Output/O.U.) proprement et clairement le    %
% problème.
%
% Il est aussi demandé de réaliser une analyse complète (i.e., découpe en SPs)           %
%%%%%%%%%%%%%%%%%%%%%%%%%%%%%%%%%%%%%%%%%%%%%%%%%%%%%%%%%%%%%%%%%%%%%%%%%%%%%%%%%%%%%%%%%%

\section{Définition et Analyse du Problème}\label{analyse}
%%%%%%%%%%%%%%%%%%%%%%%%%%%%%%%%%%%%%%%%%%%%


\subsection{Input/Output}

\begin{itemize}
\item \textbf{Input}:
    Un tableau d'entiers $T$ de taille $N$: \\
    $ T = (T[0], T[1] \dots T[N - 1]) $ \\
    $ \land $ \\
    $ N \geq 0 $

\item \textbf{Output}:
    Un entier $k$ représentant la longueur maximale des sous-tableaux (préfixe et suffixe)
    du tableau $T$. \\
    $k < N \land \operatorname{pref\_equals\_suff}(T, N, k)$ \\
    \\
    Si de tels sous-tableaux n'existent pas, renvoyez 0.
\end{itemize}


\subsection{Découpe en sous-problèmes}

\begin{itemize}
\item \textbf{SP 1}: Énumération des longueurs possibles pour les préfixes et suffixes \\
    Nous parcourons toutes les longueurs possibles $k$ (pour les préfixes et suffixes)
    du tableau $T$ dans l'ordre décroissant de $N-1$ jusqu’à $1$.
    Pour chaque longueur $k$, nous vérifions la correspondance en utilisant la méthode de comparaison (SP 2).
    Dès que nous trouvons une longueur $k$ pour laquelle le préfixe et le suffixe correspondent,
    nous la retournons. \\
    Si aucune longueur ne convient, nous retournons 0.

\item \textbf{SP 2}: Vérification de l’égalité entre préfixe et suffixe \\
    Pour une longueur $k$ donnée,
    nous vérifions si le préfixe et le suffixe de $T$ sont identiques.
    Cette comparaison s’effectue élément par élément.

\end{itemize}
