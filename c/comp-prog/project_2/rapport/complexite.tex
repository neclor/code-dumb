% !TEX root = ./main.tex
%%%%%%%%%%%%%%%%%%%%%%%%%%%%%%%%%%%%%%%%%%%%%%%%%%%%%%%%%%%%%%%%%%%%%%%%%%%%%%%%%%%%%%%%%%
% Dans cette section, vous devez étudier complètement la complexité de votre code.       %
% Soyez le plus formel (i.e., mathématique) possible.                                    %
%%%%%%%%%%%%%%%%%%%%%%%%%%%%%%%%%%%%%%%%%%%%%%%%%%%%%%%%%%%%%%%%%%%%%%%%%%%%%%%%%%%%%%%%%%
\section{Complexité}\label{complexite}
%%%%%%%%%%%%%%%%%%%%

\subsection{Fonctions sur Escale}

\begin{align*}
\texttt{escale\_create} : \mathcal{O}(1) \\
\texttt{escale\_get\_name} : \mathcal{O}(1)\\
\texttt{escale\_get\_x} : \mathcal{O}(1)\\
\texttt{escale\_get\_y} : \mathcal{O}(1)\\
\texttt{escale\_get\_best\_time} : \mathcal{O}(1)\\
\texttt{escale\_set\_best\_time} : \mathcal{O}(1)\\
\texttt{escale\_distance} : \mathcal{O}(1)\\
\texttt{escale\_equal} : \mathcal{O}(1)\\
\end{align*}

\subsection{Opérations sur Course (tableau dynamique)}
Soit $n$ le nombre d'escales, $S$ la capacité du tableau.

\begin{align*}
\texttt{course\_create} : \mathcal{O}(1)\\
\texttt{course\_is\_circuit} : \mathcal{O}(1)\\
\texttt{course\_get\_escales\_count} : \mathcal{O}(1) \\
\texttt{course\_get\_stages\_count} : \mathcal{O}(1) \\
\texttt{course\_total\_time} : \mathcal{O}(n) \\
\texttt{course\_best\_time\_at}(i) : \mathcal{O}(1) \\
\texttt{course\_append} :
    \begin{cases}
        \mathcal{O}(1) & \\
        \mathcal{O}(n) & \text{(réallocation)}
    \end{cases} \\
\texttt{course\_pop} : \mathcal{O}(1)
\end{align*}

\subsection{Opérations sur \texttt{Course} (liste chaînée)}

Soit $n$ le nombre d'escales.

\begin{align*}
\texttt{course\_create} : \mathcal{O}(1)\\
\texttt{course\_is\_circuit} : \mathcal{O}(n)\\
\texttt{course\_get\_escales\_count} : \mathcal{O}(n) \\
\texttt{course\_get\_stages\_count} : \mathcal{O}(n) \\
\texttt{course\_total\_time} : \mathcal{O}(n) \\
\texttt{course\_best\_time\_at}(i) : \mathcal{O}(i) \\
\texttt{course\_append} : \mathcal{O}(n) \\
\texttt{course\_pop} : \mathcal{O}(n)
\end{align*}
