% !TEX root = ./main.tex
%%%%%%%%%%%%%%%%%%%%%%%%%%%%%%%%%%%%%%%%%%%%%%%%%%%%%%%%%%%%%%%%%%%%%%%%%%%%%%%%%%%%%%%%%%
% Dans cette section, expliquez les structures de données mises en place pour implémenter%
% les différents TAD                                                                     %
% Pensez à discuter des avantages et inconvénients de chacune de vos structures.         %
%%%%%%%%%%%%%%%%%%%%%%%%%%%%%%%%%%%%%%%%%%%%%%%%%%%%%%%%%%%%%%%%%%%%%%%%%%%%%%%%%%%%%%%%%%
\section{Structures de Données}\label{structures}
%%%%%%%%%%%%%%%%%%%%%%%%%%%%%%%

Pour implémenter les différents TAD, nous avons choisi deux types de structures de données : le tableau dynamique et la liste chaînée.

\subsection{Escale}
\begin{lstlisting}[caption={Structure de Escale}]
    typedef struct Escale {
        char *name;
        double x;
        double y;
        double time;
    } Escale;
\end{lstlisting}

\subsection{Course (Tableau)}
\begin{lstlisting}[caption={Structure de Course (tableau)}]
    typedef struct Course {
        size_t escales_size;
        size_t escales_count;
        Escale **escales;
    } Course;
\end{lstlisting}

\subsubsection{Avantages}
\begin{itemize}
    \item Accès rapide aux éléments par leur indice ($O(1)$).
    \item Moins de surcharge mémoire due aux pointeurs supplémentaires.
    \item Facile à parcourir séquentiellement.
\end{itemize}

\subsubsection{Inconvénients}
\begin{itemize}
    \item Redimensionnement coûteux si la taille initiale est insuffisante ($O(n)$).
    \item Ajout et suppression au milieu nécessitent un déplacement des éléments ($O(n)$).
\end{itemize}

\subsection{Course (Liste Chaînée)}
\begin{lstlisting}[caption={Structure de Course (liste chainée)}]
    typedef struct Course {
        Escale *escale;
        Course *next;
    } Course;
\end{lstlisting}

\subsubsection{Avantages}
\begin{itemize}
    \item Insertion et suppression en temps constant ($O(1)$) sans déplacement des éléments.
    \item Taille flexible sans besoin de redimensionnement.
\end{itemize}

\subsubsection{Inconvénients}
\begin{itemize}
    \item Accès séquentiel aux éléments ($O(n)$) au lieu d'un accès direct.
    \item Surcharge mémoire due aux pointeurs supplémentaires.
\end{itemize}

Ces choix de structures de données permettent de répondre aux différentes exigences du problème.
Le tableau est idéal pour un accès rapide et indexé,
tandis que la liste chaînée convient mieux aux modifications
fréquentes et dynamiques de la course.

\subsection{Structure de données}

