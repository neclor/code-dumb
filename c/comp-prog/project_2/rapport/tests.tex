% !TEX root = ./main.tex
%%%%%%%%%%%%%%%%%%%%%%%%%%%%%%%%%%%%%%%%%%%%%%%%%%%%%%%%%%%%%%%%%%%%%%%%%%%%%%%%%%%%%%%%%%
% Dans cette section, décrivez comment vous avez implémenté les différents tests         %
% unitaires                                                                              %
% Pensez à justifier vos choix.                                                          %
%%%%%%%%%%%%%%%%%%%%%%%%%%%%%%%%%%%%%%%%%%%%%%%%%%%%%%%%%%%%%%%%%%%%%%%%%%%%%%%%%%%%%%%%%%
\section{Tests Unitaires}\label{tests}
%%%%%%%%%%%%%%%%%%%%%%%%%

Les tests unitaires sont situés dans le dossier \texttt{test/} et utilisent le framework \texttt{seatest}.
Pour chaque implémentation (\texttt{course\_liste} et \texttt{course\_tableau}), deux fonctions principales sont testées :

\begin{itemize}
    \item \texttt{test\_course\_append} : vérifie que l'ajout d'escales modifie correctement le nombre d'escales dans la course.
    \item \texttt{test\_course\_total\_time} : vérifie que la somme des temps des escales est correcte après modification des temps et suppression d'éléments.
\end{itemize}

Les assertions utilisées sont \texttt{assert\_ulong\_equal} pour les entiers et \texttt{assert\_double\_equal} pour les valeurs réelles.

\textbf{Justification des choix :}
Les tests ciblent les opérations principales et les cas limites.


\textbf{Limite :}
Seules les fonctions d'ajout et de calcul du temps total sont testées.
Les autres fonctions publiques ne sont pas couvertes par les tests actuels.
