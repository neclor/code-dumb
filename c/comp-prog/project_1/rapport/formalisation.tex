% !TEX root = ./main.tex
%%%%%%%%%%%%%%%%%%%%%%%%%%%%%%%%%%%%%%%%%%%%%%%%%%%%%%%%%%%%%%%%%%%%%%%%%%%%%%%%%%%%%%%%%%
% Dans cette section, introduisez toutes les notations mathématiques que vous jugez      %
% utiles à la réalisation du projet.                                                     %
%%%%%%%%%%%%%%%%%%%%%%%%%%%%%%%%%%%%%%%%%%%%%%%%%%%%%%%%%%%%%%%%%%%%%%%%%%%%%%%%%%%%%%%%%%
\section{Formalisation du Problème}\label{formalisation}
%%%%%%%%%%%%%%%%%%%%%%%%%%%%%%%%%%%


\textbf{Notations clés}:\\
    $ T[0 \dots k-1] $ — préfixe de la longueur $k$ du tableau $T$\\
    $ T[N-k \dots N-1] $ — suffixe de la longueur $k$ du tableau $T$\\
\\
\textbf{Prédicat}:\\
$ \operatorname{pref\_equals\_suff}(T, N, k) \equiv (0 < k < N) \land \forall i \in [0 \dots k - 1], T[i] = T[N - k + i]$\\
\\
\textbf{Fonction}:\\
$ \operatorname{prefixe\_suffixe}(T, N) \equiv \max\{k \mid k \in [0 \dots N - 1], \operatorname{pref\_equals\_suff}(T, N, k)\} $ \\
