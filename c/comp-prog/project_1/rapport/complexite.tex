% !TEX root = ./main.tex
%%%%%%%%%%%%%%%%%%%%%%%%%%%%%%%%%%%%%%%%%%%%%%%%%%%%%%%%%%%%%%%%%%%%%%%%%%%%%%%%%%%%%%%%%%
% Dans cette section, vous devez étudier complètement la complexité de votre code.       %
% Soyez le plus formel (i.e., mathématique) possible.                                    %
%%%%%%%%%%%%%%%%%%%%%%%%%%%%%%%%%%%%%%%%%%%%%%%%%%%%%%%%%%%%%%%%%%%%%%%%%%%%%%%%%%%%%%%%%%
\section{Complexité}\label{complexite}
%%%%%%%%%%%%%%%%%%%%

\textbf{Complexité:}

\begin{itemize}
    \item Complexité de la fonction \texttt{pref\_equal\_suff}:
        \begin{itemize}
            \item Dans le pire cas, la fonction effectue $k$ comparaisons
            \item Dans cas maximal ($k = N-1$), la complexité est $\mathcal{O}(N)$
        \end{itemize}

    \item Complexité de la fonction \texttt{prefixe\_suffixe}:
        \begin{itemize}
            \item Dans le pire cas, itère sur toutes les valeurs de $k$ de $N-1$ à $1$
            \item Appelle \texttt{pref\_equal\_suff} pour chaque $k$
        \end{itemize}

\end{itemize}

\bigskip

\textbf{Complexité totale:}
    \begin{itemize}
        \item $\sum_{k=1}^{N-1} \mathcal{O}(k) = \mathcal{O}(N^2)$
    \end{itemize}
