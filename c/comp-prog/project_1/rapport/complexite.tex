% !TEX root = ./main.tex
%%%%%%%%%%%%%%%%%%%%%%%%%%%%%%%%%%%%%%%%%%%%%%%%%%%%%%%%%%%%%%%%%%%%%%%%%%%%%%%%%%%%%%%%%%
% Dans cette section, vous devez étudier complètement la complexité de votre code.       %
% Soyez le plus formel (i.e., mathématique) possible.                                    %
%%%%%%%%%%%%%%%%%%%%%%%%%%%%%%%%%%%%%%%%%%%%%%%%%%%%%%%%%%%%%%%%%%%%%%%%%%%%%%%%%%%%%%%%%%
\section{Complexité}\label{complexite}
%%%%%%%%%%%%%%%%%%%%

\textbf{Complexité de \texttt{pref\_equal\_suff}}:
\begin{itemize}
    \item Complexité exacte:\\
        $ T_1(k) = 1 + (k+1) + k + k + 1 = 3k + 3$\\
    \item Asymptotique:\\
        $ T_1(k) \in \mathcal{O}(k) $
\end{itemize}

\textbf{Complexité de \texttt{prefixe\_suffixe}}:
\begin{itemize}
    \item Complexité exacte:\\
        \[
        T_2(N) = 1 + \sum_{k=N-1}^{1} [3 + T_1(k)] + 1
        = 2 + 3 \sum_{k=N-1}^{1} [3k + 6]
        = 2 + \frac{3(N-1)N}{2} + 6(N - 1)
        = \frac{3N^2 - 3N}{2} + 6N - 4
        \]
    \item Asymptotique:\\
        $ T_2(N) \in \mathcal{O}(N^2) $
\end{itemize}

